\section{Vollständige Induktion}
\begin{frame}
    \frametitle{Vollständige Induktion}
    Beispiel: Gausssche Summenformel

    Zu beweisende Aussage: $\sum_{i=1}^{n} i = \frac{n(n+1)}{2}$

    Induktionsanfang(IA): $n = 1$\newline
    $\sum{i=1}^{1} i = 1 = \frac{1(1+1)}{2}$

    Induktionsvoraussetzung(IV):\newline
    Die Aussage gelte für ein beliebiges aber festes $n \in \mathbb{N}$

    Induktionsschritt(IS): $n \Rightarrow n+1$\newline
    $\sum_{i=1}^{n+1} i = \sum_{i=1}^{n} i + (n+1) \stackrel{IV}{=} \frac{n(n+1)}{2} + (n+1) = \frac{n(n+1)}{2} + \frac{2n+2}{2} = \frac{n(n+1)+2n+2}{2} = \frac{(n+1)(n+2)}{2} \blacksquare$
\end{frame}
