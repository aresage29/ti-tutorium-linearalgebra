\section{Vollständige Induktion}
\begin{frame}
    \frametitle{Vollständige Induktion}
    Vollständige Induktion wird verwendet um die Gültigkeit von Aussagen für alle natürlichen Zahlen zu beweisen, die größer oder gleich einem Startwert sind
    \vfill
     Vorgehen wie bei Domino:
        \newline Bedingung zuerst für den ersten "Stein" prüfen und dann für den nächsten, welcher alle anderen umwirft
    \begin{itemize}
        
        \vfill \item Induktionsanfang(IA)($n=1$):
        \vfill \item Induktionsvoraussetzung(IV):
                \newline Die Aussage gelte für ein beliebiges aber festes $n \in \mathbb{N}$
        \vfill \item Induktionsschritt(IS)($n \Rightarrow n+1$):
        \newline Die Aussage gilt für "immer das nächste n"

    \end{itemize}
\end{frame}

\begin{frame}
    \frametitle{Vollständige Induktion}
    Zu beweisende Aussage: $\sum_{i=1}^{n} i = \frac{n(n+1)}{2}$
    \vfill
    \begin{itemize}
        \vfill \item Induktionsanfang(IA)($n=1$):
                \newline $\sum_{i=1}^{1} i = \frac{1(1+1)}{2} = 1$
        \vfill \item Induktionsvoraussetzung(IV):
                \newline Die Aussage gelte für ein beliebiges aber festes $n \in \mathbb{N}$
        \vfill \item Induktionsschritt(IS)($n \Rightarrow n+1$):
                \newline $\sum_{i=1}^{n+1} i = \sum_{i=1}^{n} i + (n+1) \stackrel{IV}{=} \frac{n(n+1)}{2} + (n+1) = \frac{n(n+1)}{2} + \frac{2n+2}{2} = \frac{n(n+1)+2n+2}{2} = \frac{(n+1)(n+2)}{2} \blacksquare$

    \end{itemize}
\end{frame}

\begin{frame}
    \frametitle{Vollständige Induktion}
    Übungsaufgabe:
        \begin{itemize}

    \vfill \item Summe über Quadratzahlen: Zeige, dass für alle natürlichen Zahlen gilt:
    \newline $\sum_{i=1}^{n} i^2 = \frac{n(n+1)(2n+1)}{6}$
     \vfill \item Summe über ungerade Zahlen: Zeige, dass für alle natürlichen Zahlen gilt:
    \newline $\sum_{i=1}^{n} (2i-1) = n^2$
    \end{itemize}

\end{frame}

\begin{frame}
    \frametitle{Vollständige Induktion}
    Lösungen zu 1:
        \begin{itemize}

    \vfill \item Induktionsanfang:
    \newline $\sum_{i=1}^{1} 1^2 = \frac{1(1+1)(2*1+1)}{6}=1$
    \newline \item Induktionsvoraussetzung:
    \newline $\sum_{i=1}^{n} i^2 = \frac{n(n+1)(2n+1)}{6}$
        \vfill \item Induktionsschritt ($n \Rightarrow n+1$):
    \newline $\sum_{i=1}^{n+1} i^2 = \frac{(n+1)(n+2)(2n+3)}{6}$
            \vfill \item Folglich gilt zu zeigen:
	\newline$\sum_{i=1}^{n} i^2 + (n+1)^2 = \frac{(n+1)(n+2)(2n+3)}{6}$
    \end{itemize}

\end{frame}

\begin{frame}
    \frametitle{Vollständige Induktion}
    Lösungen zu 1:
        \begin{itemize}
            \vfill $\sum_{i=1}^{n} i^2 + (n+1)^2 = \frac{(n+1)(n+2)(2n+3)}{6}$
            \vfill \item Durch Einsetzen der IV erhalten wir:
            \vfill $\frac{n(n+1)(2n+1)}{6} + (n+1)^2 = \frac{(n+1)(n+2)(2n+3)}{6} $
            \vfill $\frac{n(n+1)(2n+1)}{6} + \frac{((n+1)^2)6}{6} = \frac{(n+1)(n+2)(2n+3)}{6}$
            \vfill $\frac{n(n+1)(2n+1) + ((n+1)^2)6}{6} = \frac{(n+1)(n+2)(2n+3)}{6}$
            \vfill \item Aufloesen ergibt:
            \vfill $\frac{2n^3 + 9n^2 + 13n + 6}{6} = \frac{2n^3 + 9n^2 + 13n + 6}{6} \blacksquare$
        \end{itemize}

\end{frame}

\begin{frame}
    \frametitle{Vollständige Induktion}
    Lösungen zu 2:
        \begin{itemize}
            \vfill \item Induktionsanfang:
            \vfill \newline $\sum_{i=1}^{1} 2*1-1 = 1^2$
            \vfill \item Induktionsvoraussetzung:
            \vfill \newline $\sum_{i=1}^{n} (2*i-1) = n^2$
            \vfill \item Induktionsschritt:
            \vfill \newline $\sum_{i=1}^{n+1} (2*i-1) = (n+1)^2$
            \vfill \newline $\sum_{i=1}^{n} (2*i-1)+(2(n+1)-1) = (n+1)^2$
            \vfill \newline $\sum_{i=1}^{n} (2*i-1)+(2n+1) = (n+1)^2$
        \end{itemize}

\end{frame}

\begin{frame}
    \frametitle{Vollständige Induktion}
    Lösungen zu 2:
        \begin{itemize}
            \vfill \item Durch Einsetzen der IV erhalten wir:
            \vfill \newline $n^2+(2n+1) = (n+1)^2$
            \vfill \item Aufloesen ergibt:
            \vfill \newline $n^2+2n+1 = n^2 + 2n + 1 \blacksquare$
        \end{itemize}
\end{frame}