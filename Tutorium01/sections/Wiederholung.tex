\section{Wiederholung}
\begin{frame}
    \frametitle{Bruchgesetze}
    \begin{enumerate}
        \vfill \item $\frac{a}{b}=\frac{ac}{bc}$
        \vfill \item $\frac{a}{b}=\frac{a \div c}{b \div c}$
        \vfill \item $\frac{a}{b}*c=\frac{ac}{b}$
        \vfill \item $\frac{a}{b}*c^{-1}=\frac{a}{bc}$
        \vfill \item $\frac{a}{b} \div c=\frac{a}{bc}$
    \end{enumerate}
\end{frame}

\begin{frame}
    \frametitle{Bruchgesetze Rechenarten}
    \begin{enumerate}
        \vfill \item $\frac{a}{b} + \frac{c}{b}=\frac{a+c}{b}$
        \vfill \item $\frac{a}{b} - \frac{c}{b}=\frac{a-c}{b}$
        \vfill \item $\frac{a}{b} + \frac{c}{d}=\frac{ad+bc}{bd}$
        \vfill \item $\frac{a}{b} = \frac{1}{\frac{b}{a}}$
        \vfill \item $\frac{a}{b} \div \frac{c}{d} = \frac{\frac{a}{b}}{\frac{c}{d}} = \frac{ad}{bc}$
    \end{enumerate}
\end{frame}


\begin{frame}
    \frametitle{Bruchgesetzte Formeln Übungen}
    \begin{enumerate}
	\vfill \item $\frac{a}{b} * (1 + \frac{c}{1-c})$
        \vfill \item $\frac{a}{\ln(1)}$
        \vfill \item $\frac{6p + 11}{2p + 4} - \frac{2p + 5}{p^2 +2p} -3$
	   \vfill \item $\frac{1}{x - y} - \frac{1}{x + y} $
    \end{enumerate}
\end{frame}

\begin{frame}
    \frametitle{Bruchgesetze Formeln Übungen Lösungen}
	Lösungen für Aufgabe 1 und 2:
    \begin{enumerate}
	\vfill \item $\frac{a}{b} * (1 + \frac{c}{1-c}) = \frac{a}{b}*(\frac{1-c}{1-c}+\frac{c}{1-c}) = \frac{a}{b}*(\frac{1}{1-c})$
        \vfill \item $\frac{a}{\ln(1)}$ Nicht erlaubt, da durch 0 geteilt wird
    \end{enumerate}
\end{frame}

\begin{frame}
    \frametitle{Bruchgesetze Formeln Übungen Lösungen}
	Lösungen für Aufgabe 3:
	\begin{itemize}
	\vfill \item $\frac{6p + 11}{2p + 4} - \frac{2p + 5}{p^2 +2p} -3 = \frac{6p^2 + 11p}{2p^2 + 4p} - \frac{4p + 5p}{2p^2 + 4p} - \frac{6p^2 + 12p}{2p^2 + 4p}$
	\vfill \item $\frac{6p^2 + 11p}{2p^2 + 4p} - \frac{4p + 10}{2p^2 + 4p} - \frac{6p^2 + 12p}{2p^2 + 4p} = \frac{-5p - 10}{2p^2 + 4p}$
	\vfill \item$ \frac{-5p - 10}{2p^2 + 4p} = \frac{-0.5 * (2 * (5p + 10))}{0.2p * (5 * (2p + 4))}$
	\vfill \item$\frac{-0.5 * (2 * (5p + 10))}{0.2p * (5 * (2p + 4))} = \frac{-0.5}{0.2p} = -\frac{5}{2p}$
	\end{itemize}
\end{frame}

\begin{frame}
    \frametitle{Bruchgesetze Formeln Übungen Lösungen}
	Lösungen für Aufgabe 4:
	\begin{itemize}
	\vfill \item$\frac{1}{x - y} - \frac{1}{x + y} = \frac{2y}{x^2-y^2}$
	\end{itemize}
\end{frame}

\begin{frame}
    \frametitle{Binomische Formeln}
    \begin{enumerate}
        \vfill \item $(a+b)^2 = a^2 + 2ab + b^2$
        \vfill \item $(a-b)^2 = a^2 - 2ab + b^2$
        \vfill \item $(a+b)(a-b) = a^2 - b^2$
    \end{enumerate}
\end{frame}

\begin{frame}
    \frametitle{Binomische Formeln Übungen}
    \begin{enumerate}
        \vfill \item $(2+r)^2 - (2-r)^2$
        \vfill \item $\frac{1}{2}a^2 - 4ab + 4b^2$
        \vfill \item $(0.5-y)^2 - (0.5x - y)*(0.5 + y)$
	\vfill \item $(0.5-y)^2 - (0.5x - y)*(0.5x + y)$
    \end{enumerate}
\end{frame}

\begin{frame}
    \frametitle{Binomische Formeln Übungen Lösungen}
    \begin{enumerate}
        \vfill \item $(2+r)^2 - (2-r)^2 = (4 + 4r +r^2) - (4 - 4r + r^2) = 8r$
        \vfill \item $\frac{1}{2}a^2 - 4ab + 4b^2$ hier kann keine binomische Formel angewendet werden
        \vfill \item $(0.5-y)^2 - (0.5x - y)*(0.5 + y) = 2y^2 -0.5y-0.25x-0.5xy+0.25$
	\vfill \item $(0.5-y)^2 - (0.5x - y)*(0.5x + y) =
	\newline -y - 0,25x^2 + 0,25$
    \end{enumerate}
\end{frame}


\begin{frame}
    \frametitle{Potenzgesetze}
	\begin{itemize}
		\vfill \item $a^m * a^n = a^{m+n}$
		\vfill \item $\frac{a^m}{a^n} = a^{m-n}$
		\vfill \item $a^n * b^n = (ab)^n$
		\vfill \item $\frac{a^n}{b^n} = \frac{a}{b}^n$
		\vfill \item $(a^m)^n = a^{m*n}$
		\vfill \item $a^0 = 1$
		\vfill \item $a^{-n} = \frac{1}{a^n}$
	\end{itemize}
\end{frame}

\begin{frame}
    \frametitle{Potenzgesetze Übungen}
	\begin{enumerate}
		\vfill \item $\frac{1^3}{5^n}$
		\vfill \item $\frac{5^0- \sin(a^2) + \cos(a^2)}{\ln(e)}$
		\vfill \item $\frac{5^0-( \sin(a)^2 + \cos(a)^2)}{\ln(e)}$
		\vfill \item $x^{-n} * x$
		\vfill \item $\frac{x^5 +1}{x^{m+2}} - \frac{2x^2-2}{x^m} + \frac{2-x}{x^{m-2}}$
	\end{enumerate}
\end{frame}

\begin{frame}
    \frametitle{Potenzgesetze Lösungen}
	\begin{enumerate}
		\vfill \item $\frac{1^3}{5^n} =  (\frac{1}{5^n}) = (\frac{1}{5})^n$
		\vfill \item $\frac{5^0- \sin(a^2) + \cos(a^2)}{\ln(e)} = \frac{1 - \sin(a^2) + \cos(a^2)}{1} = 1 - \sin(a^2) + \cos(a^2)$
		\vfill \item $\frac{5^0-( \sin(a)^2 + \cos(a)^2)}{\ln(e)} = \frac{1-1}{1} = 0$
		\vfill \item $x^{-n} * x = x^{-n+1}$
		\vfill \item $\frac{x^5 +1}{x^{m+2}} - \frac{2x^2-2}{x^m} + \frac{2-x}{x^{m-2}} = \frac{x^5+1}{x^{m+2}} - \frac{2x^2*x^2-2*x^2}{x^{m+2}} + \frac{2*x^4-x*x^4}{x^{m+2}}= \frac{x^5+1-2x^4+2x^2+2x^4-x^5}{x^{m+2}} = \frac{1+2x^2}{x^{m+2}}$
	\end{enumerate}
\end{frame}
