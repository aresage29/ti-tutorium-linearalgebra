\section{Rang Kern Bild}
\subsection{Rang}
\begin{frame}
    \frametitle{Rang}
    \begin{itemize}
        \item $rang(A) \Leftrightarrow$ ist die Anzahl linear unabhängiger Zeilen bzw. Spalten (Wegen $rang(A)=rang(A^\top)$)
        \item Elementare Zeilenumformungen verändern den Rang nicht $\Rightarrow$ in ZSF sind Zeilen mit führender 1 linear unabhängig
        \item Rang eines LGS = Rang der Koeffizientenmatrix
        \item LGS lösbar $\Leftrightarrow$ Rang der Koeffizientenmatrix = Rang der Erweiterten Matrix
    \end{itemize}
\end{frame}

\begin{frame}
	\frametitle{Beispiel}
	\begin{align*}
		A = \begin{pmatrix}
			2 & 4 & 0 \\
			0 & 0 & 1 \\
			0 & 0 & 0
		\end{pmatrix}
	\end{align*}
	\begin{itemize}
		\item $rang(A) = 2$
		\item $A$ ist schon fast in Zeilenstufenform und somit kann der Rang leicht abgelesen werden
	\end{itemize}
\end{frame}

\begin{frame}
	\frametitle{Übung I}
	Bestimme den Rang der Matrix
	\begin{align*}
		\begin{pmatrix}
			1 & -2 & 1 \\
			3 & -2 & -4 \\
			2 & 6 & -4 \\
			1 & 3 & -2 
		\end{pmatrix}
	\end{align*}
\end{frame}

\begin{frame}
	\frametitle{Übung I - Lösung}
	Bestimme den Rang der Matrix
	\begin{align*}
		\begin{pmatrix}
			1 & -2 & 1 \\
			3 & -2 & -4 \\
			2 & 6 & -4 \\
			1 & 3 & -2 
		\end{pmatrix} \\
		\leftrightarrow
		\begin{pmatrix}
			1 & 0 & 0 \\
			0 & 1 & 0 \\
			0 & 0 & 1 \\
			0 & 0 & 0
		\end{pmatrix} 
	\end{align*}
	$\rightarrow$ der Rang von $A$ ist somit $3$.
\end{frame}


\subsection{Bild}
\begin{frame}
    \frametitle{Bild}
    \begin{itemize}
        \item $Bild(A)={Ax | x\in \mathbb{R}^n}$
        \item Menge aller Elemente, die durch Anwendung der Abbildung $A$ auf den Vektorraum $\mathbb{R}^n$ entstehen
    \end{itemize}
\end{frame}

\begin{frame}
    \frametitle{Bild Berechnung Verfahren 1}
    \begin{enumerate}
        \item Matrix transponieren
		\item Matrix in Zeilenstufenform umwandeln
		\item Matrix zurück transponieren
		\item Lösung aufschreiben
    \end{enumerate}
\end{frame}

\begin{frame}
    \frametitle{Bild Berechnung Verfahren 2}
    \begin{itemize}
        \item Doppeltes Transponieren vermeiden
		\item Matrix in untere Dreiecksmatrix umwandeln (Achtung: Statt Zeilenumformungen Spaltenumformungen verwenden)
    \end{itemize}
\end{frame}

\begin{frame}
    \frametitle{Übung}
	Es sei
    $A =
    \begin{pmatrix}
        1 & 3 & 2 \\
        2 & 4 & 4 \\
        3 & 5 & 6
    \end{pmatrix}$
    \begin{enumerate}
        \item Geben Sie das Bild von $A$ an.
    \end{enumerate}
\end{frame}

\begin{frame}
    \frametitle{Übung - Lösung}
	Es sei
    $A =
    \begin{pmatrix}
        1 & 3 & 2 \\
        2 & 4 & 4 \\
        3 & 5 & 6
    \end{pmatrix}$
    \begin{enumerate}
		\item Matrix transponieren $\Rightarrow
		\begin{pmatrix}
			1 & 2 & 3 \\
			3 & 4 & 5 \\
			2 & 4 & 6
		\end{pmatrix}$
		\item Matrix in Zeilenstufenform bringen $\Rightarrow
		\begin{pmatrix}
			1 & 2 & 3 \\
			0 & -2 & -4 \\
			0 & 0 & 1
		\end{pmatrix}$
		\item Matrix zurück transponieren $\Rightarrow
		\begin{pmatrix}
			1 & 0 & 0 \\
			2 & -2 & 0 \\
			3 & -4 & 0
		\end{pmatrix}$
    \end{enumerate}
\end{frame}

\begin{frame}
    \frametitle{Übung - Lösung}
    \begin{enumerate}
		\setcounter{enumi}{4}
        \item
		$Bild(A) = {
		\lambda_1 \cdot
		\begin{pmatrix}
			1 \\
			2 \\
			3
		\end{pmatrix}
		+
		\lambda_2 \cdot
		\begin{pmatrix}
			0 \\
			-2 \\
			-4
		\end{pmatrix}
		| \lambda_1, \lambda_2 \in \mathbb{R}
		}$
		\item Oder: $Bild(A) = \Biggl\langle \begin{pmatrix}1 \\ 2 \\ 3\end{pmatrix}, \begin{pmatrix}0 \\ -2 \\ -4\end{pmatrix} \Biggr\rangle$ (auch lineare Hülle oder Spann genannt)
		\item Zusatz: $\dim(Bild(A)) = 2$ gilt $Rang(A)=2$
    \end{enumerate}
\end{frame}

\subsection{Kern}
\begin{frame}
    \frametitle{Kern}
    \begin{itemize}
        \item $Ker(A)={x | Ax=0}$
        \item Menge aller Elemente, die auf 0 abgebildet werden
        \item $Ker(A)$ ist Lösungsmenge des zu A gehörenden homogenen LGS
    \end{itemize}
\end{frame}

\begin{frame}
	\frametitle{Beispiel}
	\begin{align*}
		A = \begin{pmatrix}
			1 & 1 & 2 \\
			0 & 1 & 1 \\
			1 & 0 & 1 
		\end{pmatrix}
	\end{align*}
	Um den Kern zu bestimmen muss das homogene LGS Ax = 0 gelöst werden.
	\begin{align*}
		\begin{pmatrix}
			1 & 1 & 2 \\
			0 & 1 & 1 \\
			1 & 0 & 1 
		\end{pmatrix} \rightarrow
		\begin{pmatrix}
			1 & 1 & 2 \\
			0 & 1 & 1 \\
			0 & -1 & -1 
		\end{pmatrix} \rightarrow
		\begin{pmatrix}
			1 & 0 & 1 \\
			0 & 1 & 1 \\
			0 & 0 & 0 
		\end{pmatrix} 
	\end{align*}
\end{frame}

\begin{frame}
	\frametitle{Beispiel}
	\begin{align*}
		a + c = 0 \\
		b + c = 0 \\
	\end{align*}
	Wir definieren dann zum Beispiel $c = t$ und erhalten
	\begin{align*}
		a = -t \\
		b = -t
	\end{align*}
	Somit kommen wir dann auf Ker(A) =
	$\{t \cdot (-1, -1, 1) | t \in \mathbb{R}\}$
\end{frame}

\begin{frame}
	\frametitle{Beispiel}
	Da wir die Matrix A den Rang 2 hat haben wir 2 unabhängige Spaltenvektoren.
	Daraus folgt das Bild:
	\begin{align*}
		Bild(A) = \{k_1 \cdot \begin{pmatrix}
			1 \\
			0 \\
			1
		\end{pmatrix} + k_2 \cdot \begin{pmatrix}
			1 \\
			1 \\
			0
		\end{pmatrix} | k_1, k_2 \in \mathbb{R}\}
	\end{align*}
\end{frame}

\begin{frame}
    \frametitle{Übung}
    Es sei
    $A =
    \begin{pmatrix}
        1 & 2 & 3 \\
        4 & 5 & 6 \\
        7 & 8 & 9
    \end{pmatrix}$
    \begin{enumerate}
        \item Bestimmen Sie den Kern von $A$
    \end{enumerate}
\end{frame}

\begin{frame}
    \frametitle{Übung - Lösung}
    Es sei
    $A =
    \begin{pmatrix}
        1 & 2 & 3 \\
        4 & 5 & 6 \\
        7 & 8 & 9
    \end{pmatrix}$
    \begin{itemize}
        \item $|A| = 0 \Rightarrow$ $A$ besitzt einen Kern
        \item $Kern(A) =
        \begin{pmatrix}
            1 & 2 & 3 \\
            4 & 5 & 6 \\
            7 & 8 & 9
        \end{pmatrix}
        \cdot
        \begin{pmatrix}
            v_1 \\
            v_2 \\
            v_3
        \end{pmatrix}
        =
        \begin{pmatrix}
            0 \\
            0 \\
            0
        \end{pmatrix}$
        \item $\rightarrow{Gauss} \begin{pmatrix}
            1 & 2 & 3 \\
            0 & -3 & -6 \\
            0 & 0 & 0
        \end{pmatrix} 
		
		\Rightarrow$ $v_1 + 2v_2 + 3v_3 = 0$ und $-3v_2 -6v3 = 0$
		\item also ein LGS mit 2 Gleidchungen und 3 Unbekannten $\rightarrow$ unendlich viele Lösungen
    \end{itemize}
\end{frame}

\begin{frame}
    \frametitle{Übung - Lösung}
	$v_1 + 2v_2 + 3v_3 = 0$ \\
	$-3v_2 -6v3 = 0$
    \begin{itemize}
        \item Wir definieren $v_3 = t$
		\item $-3v_2 -6t = 0 \Rightarrow v_2 = \frac{-6}{3}t = -2t$
		\item $v_1 + 2 \cdot (-2)t + 3t = 0 \Rightarrow v_1 = 4t - 3t = t$
		\item $Kern(A) = \{t \cdot \begin{pmatrix}1 \\ -2 \\ 1 \end{pmatrix} | t \in \mathbb{R}\}$
    \end{itemize}
\end{frame}