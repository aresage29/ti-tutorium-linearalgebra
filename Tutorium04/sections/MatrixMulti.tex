\section{Matrix Multiplikation}
\subsection{Grundlagen}
\begin{frame}
\frametitle{Matrix Multiplikation}
Wenn die \textbf{Anzahl der Spalten} der ersten Matrix \textbf{gleich} der \textbf{Anzahl der Zeilen} der zweiten Matrix, können diese Multipliziert werden.

$c_{ij} = \sum_{k=1}^n a_{ik}b_{kj}= a_{i1}b_{1j} + a_{i2}b_{2j} + \dots + a_{in}b_{nj}$

$ A = \begin{pmatrix} 2 & 1 & 0 \\ -1 & -5 & 4 \end{pmatrix} $,
$ B = \begin{pmatrix} 3 & 5 \\ -2 & 1 \\ 0 & 7 \end{pmatrix} $
$
\begin{array}{c c c|c c}
& & & 3 & 5\\
& & & -2 & 1\\
& & & 0 & 7\\
\hline
2 & 1 & 0 & 4 & 11\\
-1 & -5 & 4 & 7 & 18\\
\end{array}
$
$ A \cdot B = \begin{pmatrix} 4 & 11 \\ 7 & 18 \end{pmatrix} $
\end{frame}

\begin{frame}
\frametitle{Matrix Multiplikation - Rechenregeln}
\textbf{Wichtig: Kommutativgesetz gilt nicht!}\\
Rechenregeln:
\begin{itemize}
\item $(kA)B = k(AB) = A(kB)$
\item $A(BC) = (AB)C$ (Assoziativgesetz)
\item $(A+B)C = AC + BC$ (Distributivgesetz 1)
\item $A(B+C) = AB + AC$ (Distributivgesetz 2)
\item $(AB)^T = B^T \cdot A^T$
\item $\mathbb{I}_n\cdot A = A\cdot \mathbb{I}_n = A$
\end{itemize}
\end{frame}

\begin{frame}
\frametitle{Matrix Multiplikation - Übungen}
\begin{enumerate}
\item $ \begin{pmatrix} 
1 & 4 & 0 & 1\\ 
0 & 2 & 1 & 3 
\end{pmatrix} 
\cdot 
\begin{pmatrix} 
1 & 0 & 3 \\ 
0 & 1 & 1\\
3 & 4 & 1\\
2 & 0 & 1
\end{pmatrix} $
\item 
$\begin{pmatrix} 
1 & 0 & 3 \\ 
0 & 1 & 1\\
3 & 4 & 1\\
2 & 0 & 1
\end{pmatrix} 
\cdot 
\begin{pmatrix} 
1 & 4 & 0 & 1\\ 
0 & 2 & 1 & 3 
\end{pmatrix} $
\item
$\begin{pmatrix}
2 & -3.1 \\
2 & 1.6 \\
\end{pmatrix}
\cdot
\begin{pmatrix}
-0.7 \\
-4.5 
\end{pmatrix}$
\end{enumerate}
\end{frame}

\begin{frame}
\frametitle{Matrix Multiplikation - Übungen Lösung}
\begin{enumerate}
\item $ \begin{pmatrix} 
1 & 4 & 0 & 1\\ 
0 & 2 & 1 & 3 
\end{pmatrix} 
\cdot 
\begin{pmatrix} 
1 & 0 & 3 \\ 
0 & 1 & 1\\
3 & 4 & 1\\
2 & 0 & 1
\end{pmatrix} 
= 
\begin{pmatrix} 
3 & 4 & 8 \\ 
9 & 6 & 6
\end{pmatrix} $
\item 
$\begin{pmatrix} 
1 & 0 & 3 \\ 
0 & 1 & 1\\
3 & 4 & 1\\
2 & 0 & 1
\end{pmatrix} 
\cdot 
\begin{pmatrix} 
1 & 4 & 0 & 1\\ 
0 & 2 & 1 & 3 
\end{pmatrix} 
=> $Geht nicht.
\item
$\begin{pmatrix}
2 & -3.1 \\
2 & 1.6 \\
\end{pmatrix}
\cdot
\begin{pmatrix}
-0.7 \\
-4.5 
\end{pmatrix}
=
\begin{pmatrix}
12.55 \\
-8.6
\end{pmatrix}$
\end{enumerate}
\end{frame}

\subsection{Inverse}
\begin{frame}
\frametitle{Matrix Multiplikation - Inverse}
Nicht jede Matrix besitzt eine Inverse. Es muss auf jeden Fall eine quadratische Matrix vorliegen.\\
Nicht jede quadratische Matrix ist invertierbar.\\
$A\cdot A^{-1} = A^{-1}\cdot A = \mathbb{I}$

Wenn eine Matrix Invertierbar ist, nennt man dies \textit{invertierbar} oder \textbf{regulär}.

Regeln:
\begin{itemize}
\item $(AB)^{-1} = A^{-1}B^{-1}$
\item $(kA)^{-1} = k^{-1}A^{-1}$ k Skalar mit $k\neq0$
\item $(A^T)^{-1} = (A^{-1})^T$
\item $A^{-1} = (A^{-1})^{-1}$
\end{itemize}
\end{frame}

\begin{frame}
\frametitle{Matrix Multiplikation - Inverse bestimmen}
Schneller Test ob eine Inverse einer quadratischen Matrix existiert: det(A)$\neq$0\\
Berechnen der Inversen:
\begin{enumerate}
\item Determinante Prüfen
\item Matrix neben Trennstrich um Einheitsmatrix erweitern
\item Umformung mittels z. B. Gauß-Jordan-Algorithmus um Einheitsmatrix anstelle Ursprungsmatrix zu erzeugen
\item Matrix rechts vom Trennstrich ist jetzt die invertierte Matrix
\end{enumerate}
\end{frame}

\begin{frame} 
\frametitle{Matrix Multiplikation - Inverse bestimmen}
Sonderfall 2x2 Matrix:\\

$\begin{pmatrix}
a & b\\
c & d
\end{pmatrix}^{-1} = 
\frac{1}{ad - bc} \begin{pmatrix}
d & -b\\
-c & a
\end{pmatrix}$
\end{frame}

\begin{frame}
\frametitle{Matrix Multiplikation - Inverse - Übungen}
\begin{enumerate}
\item Ist Matrix $A = \begin{pmatrix} 2 & 4 \\ 4 & 2 \end{pmatrix}$ regulär? Wenn ja dann gebe die Inverse an.
\item Ist Matrix $B = \begin{pmatrix} 1 & 2 & 3 \\ 4 & 5 & 6 \end{pmatrix}$ regulär? Wenn ja dann gebe die Inverse an.
\item Ist Matrix $C = \begin{pmatrix} 12 & 4 \\ 0 & 0  \end{pmatrix}$ regulär? Wenn ja dann gebe die Inverse an.
\end{enumerate}
\end{frame}

\begin{frame}
\frametitle{Matrix Multiplikation - Inverse - Übungen Lösung}
\begin{enumerate}
\item Ist Matrix $A = \begin{pmatrix} 2 & 4 \\ 4 & 2 \end{pmatrix}$ regulär? Wenn ja dann gebe die Inverse an. \\
Lösung: Ja; Gesucht Matrix $B = \begin{pmatrix} a & b \\ c & d \end{pmatrix}$
$B = \frac{1}{-12}\begin{pmatrix} 2 & -4 \\ -4 & 2 \end{pmatrix}$
\item Ist Matrix $B = \begin{pmatrix} 1 & 2 & 3 \\ 4 & 5 & 6 \end{pmatrix}$ regulär? Nein, da nicht quadratisch
\item  Ist Matrix $C = \begin{pmatrix} 12 & 4 \\ 0 & 0  \end{pmatrix}$ regulär?
Lösung: Nein denn Determinante $\neq 0$.
\end{enumerate}
\end{frame}

